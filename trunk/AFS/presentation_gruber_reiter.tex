% This text is proprietary.
% It's a part of presentation made by myself.
% It may not used commercial.
% The noncommercial use such as private and study is free
% Nov. 2006
% Author: Sascha Frank 
% University Freiburg 
% www.informatik.uni-freiburg.de/~frank/
%
% additional usepackage{beamerthemeshadow} is used
%  
%  \beamersetuncovermixins{\opaqueness<1>{25}}{\opaqueness<2->{15}}
%  with this the elements which were coming soon were only hinted
\documentclass[xcolor=dvipsnames]{beamer}
\usepackage{beamerthemeshadow}
\usepackage[ngerman]{babel}
%\usetheme{Copenhagen}
\usetheme{CambridgeUS}

\usecolortheme[named=BrickRed]{structure}
\useinnertheme{rounded}
\setbeamercovered{transparent}

\begin{document}
\title{Regulated Array Grammars of Finite Index}  
\author{C. Gruber, J. Reiter}
\institute{TU Wien}
%\date{\today} 
\date{3.M\"arz, 2010}

\frame{\titlepage} 

\frame{\frametitle{Table of contents}\tableofcontents} 


\section{Preliminaries} 
\subsection{n-dimensional array}
\frame{\frametitle{n-dimensional array} 
Ein n-dimensionales Array $A$  \"uber ein Alphabet $V$ (Menge aller non-terminal und terminal Symbole) ist eine Funktion 

\begin{eqnarray*}
A : Z^n \rightarrow V \cup \{\#\} & n \in N = \{1,2,...\}
\end{eqnarray*}

wobei 
\[ shape(A) = \{ v \in Z^n | A(v) \not = \# \}
\]
endlich ist und $\# \not \in  V$ als \textit{background} oder \textit{blank Symbol} bezeichnet wird. Das Array $A$ kann nun so definiert werden
\[ A = \{ (v,A(v)) \ | \ v \in shape(A) \}.
\]
}

\frame{\frametitle{n-dimensional array production and grammar} 
Eine n-dimensionale Array Produktion $p$ \"uber dem Alphabet $V$ ist ein Tripel $(W,A_1,A_2)$ wobei $W \subseteq Z^n$ eine endliche Menge von Koordinaten ist und $A_1$ und $A_2$ Abbildungen von $W$ auf $V \cup \{\#\}$ sind.
\newline
\newline
Eine n-dimensionale Array Grammatik kann nun als Sechstupel 
\[ G = (n,V_N,V_T,\#,P,\{(v_0,S)\})
\]
definiert werden. $\{(v_0,S)\}$ wird als Startarray (Axiom), $v_0$ als Startvektor und $S$ als das Startsymbol bezeichnet.
}

\section{Control mechanisms} 
\frame{\frametitle{matrix grammar} 
Eine Matrixgrammatik $G_M$ ist ein 4-Tupel

\[ G_M = (V_N,V_T,(M,F),S),
\]
$F$ kann auch als Fehlermenge bezeichnet werden. Ist $F= \emptyset$, dann kann $G_M$ als Matrixgrammatik ohne appearence checking bezeichnet werden.
}

\frame{\frametitle{graph controlled grammar} 
Eine graph-controlled Grammatik $G_P$ ist ein 4-Tupel

\[ G_M = (V_N,V_T,(R,L_{in},L_{fin}),S),
\]
$R$ ist eine endliche Menge von Regeln $r$ der Form 
\[ (l(r) : p(l(r)), \sigma (l(r)), \varphi (l(r))).
\]
Falls alle Felder $\varphi (l(r))$ leer sind f�r alle $r \in R$, dann kann $G_P$ als graph-controlled Grammatik ohne appearence checking bezeichnet werden.

\pause

\begin{alertblock}{}
Matrix- und graph-controlled Grammatiken k�nnen in Arraygrammatiken direkt �bergef�hrt werden, indem ihre Produktionen durch Arrayproduktionen ersetzt werden.
\end{alertblock}
}

\frame{\frametitle{bounded derivations} 
\begin{alertblock}{Index einer Ableitung}
Der Index einer Ableitung $D$ eines Terminalobjekts $w$ in einer Grammatik $G$ ist mit der maximalen Anzahl von non-terminal Symbolen, die in einem Zwischenableitungsschritt vorkommen, definiert und wird mit $ind_{G,D}(w)$ bezeichnet.
\end{alertblock}
Weiters bezeichnet $ind_{G,min}(w)$ bzw. $ind_{G,max}(w)$ das Minimum bzw. das Maximum aus der Menge

\[ \{ind_{G,D}(w) \ | \ w \ \mathrm{is \ generated \ by} \ G \}
\]
Entsprechend gibt es nun die Definition f�r die Grammatik
\begin{eqnarray*}
ind_{Y}(G) = sup \{ind_{G,Y}(w)\ | \ w \ \mathrm{is \ generated \ by} \ G \} & Y \in \{min,max\}
\end{eqnarray*}
}


\frame{\frametitle{Beispiel f�r eine Grammatik mit prescribed teams}
\textbf{2-dimensionale Array-Grammatik mit prescibed teams:}

\[ G=(n,\left\{ D,E,L,Q,R,S,U \right\} , \left\{ a \right\} , \# , (P,R,F), \left\{ ((0,0),S) \right\} )
\]

\[P= \left\{
\begin{array}{ll}
\# & \\
S & \#
\end{array}
\rightarrow
\begin{array}{ll}
L & \\
a & D
\end{array}
,
\begin{array}{l}
\# \\
L 
\end{array}
\rightarrow
\begin{array}{l}
L \\
a 
\end{array}
,
\begin{array}{ll}
D & \# \\
\end{array}
\rightarrow
\begin{array}{ll}
a & D \\
\end{array} \right
,\]
\[
\begin{array}{ll}
\# & \# \\
L &
\end{array}
\rightarrow
\begin{array}{ll}
a & U \\
a &
\end{array}
,
\begin{array}{ll}
D & \# \\
\end{array}
\rightarrow
\begin{array}{ll}
a & R \\
\end{array}
,\ \ \ \ \ \ \ \ 
\]
\[\ \ \ \ \ \ \ \ \ \ 
\left
\begin{array}{ll}
U & \# \\
\end{array}
\rightarrow
\begin{array}{ll}
a & U \\
\end{array}
,
U \rightarrow E,
\begin{array}{l}
\# \\
R
\end{array}
\rightarrow
\begin{array}{l}
Q \\
a
\end{array}
,
R \rightarrow a,
E \rightarrow a
\right\} 
\]
}

\section{Section no. 2} 
\subsection{Lists I}
\frame{\frametitle{unnumbered lists}
\begin{itemize}
\item Introduction to  \LaTeX  
\item Course 2 
\item Termpapers and presentations with \LaTeX 
\item Beamer class
\end{itemize} 
}

\frame{\frametitle{lists with pause}
\begin{itemize}
\item Introduction to  \LaTeX \pause 
\item Course 2 \pause 
\item Termpapers and presentations with \LaTeX \pause 
\item Beamer class
\end{itemize} 
}

\subsection{Lists II}
\frame{\frametitle{numbered lists}
\begin{enumerate}
\item Introduction to  \LaTeX  
\item Course 2 
\item Termpapers and presentations with \LaTeX 
\item Beamer class
\end{enumerate}
}
\frame{\frametitle{numbered lists with pause}
\begin{enumerate}
\item Introduction to  \LaTeX \pause 
\item Course 2 \pause 
\item Termpapers and presentations with \LaTeX \pause 
\item Beamer class
\end{enumerate}
}

\section{Section no.3} 
\subsection{Tables}
\frame{\frametitle{Tables}
\begin{tabular}{|c|c|c|}
\hline
\textbf{Date} & \textbf{Instructor} & \textbf{Title} \\
\hline
WS 04/05 & Sascha Frank & First steps with  \LaTeX  \\
\hline
SS 05 & Sascha Frank & \LaTeX \ Course serial \\
\hline
\end{tabular}}


\frame{\frametitle{Tables with pause}
\begin{tabular}{c c c}
A & B & C \\ 
\pause 
1 & 2 & 3 \\  
\pause 
A & B & C \\ 
\end{tabular} }


\section{Section no. 4}
\subsection{blocs}
\frame{\frametitle{blocs}

\begin{block}{title of the bloc}
bloc text
\end{block}

\begin{exampleblock}{title of the bloc}
bloc text
\end{exampleblock}


\begin{alertblock}{title of the bloc}
bloc text
\end{alertblock}
}

\section{Section no. 5}
\subsection{split screen}

\frame{\frametitle{splitting screen}
\begin{columns}
\begin{column}{5cm}
\begin{itemize}
\item Beamer 
\item Beamer Class 
\item Beamer Class Latex 
\end{itemize}
\end{column}
\begin{column}{5cm}
\begin{tabular}{|c|c|}
\hline
\textbf{Instructor} & \textbf{Title} \\
\hline
Sascha Frank &  \LaTeX \ Course 1 \\
\hline
Sascha Frank &  Course serial  \\
\hline
\end{tabular}
\end{column}
\end{columns}
}

%\subsection{Pictures} 
%\frame{\frametitle{pictures in latex beamer class}

%\begin{figure}
%\includegraphics[scale=0.5]{PIC1} 
%\caption{show an example picture}
%\end{figure}}

%\subsection{joining picture and lists} 

%\frame{
%\frametitle{pictures and lists in beamer class}
%\begin{columns}
%\begin{column}{5cm}
%\begin{itemize}
%\item<1-> subject 1
%\item<3-> subject 2
%\item<5-> subject 3
%\end{itemize}
%\vspace{3cm} 
%\end{column}
%\begin{column}{5cm}
%\begin{overprint}
%\includegraphics<2>{PIC1}
%\includegraphics<4>{PIC2}
%\includegraphics<6>{PIC3}
%\end{overprint}
%\end{column}
%\end{columns}}

\end{document}