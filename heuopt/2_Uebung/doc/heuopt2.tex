\documentclass[a4paper,10pt]{article}

\usepackage{ngerman}

%opening
\title{Heuristische Optimierungsverfahren \\ \bigskip \textbf{2.Programmier\"{u}bung}}
\author{Johannes Reiter - 0625101, Christian Gruber - 0625102}
\date{26.J\"{a}nner 2009}

\begin{document}

\maketitle

\section{Minimum Energy Broadcast Problem}
Das \textit{Minimum Energy Broadcast Problem (MEBP)} ist ein NP-hartes Optimierungsproblem f\"{u}r das Verbinden von mobilen Ger\"{a}ten in einem drahtlosen Ad-hoc Netzwerk. Es gilt eine Konfiguration f\"{u}r das Netzwerk zu finden, bei der die Sendeleitung und die n\"{o}tige Energie f\"{u}r die Kommunikation m\"{o}glichst minimal ist. Die Aufgabenstellung ist nun so, dass eine Nachricht ausgehend von einem vorgegebenen Knoten an alle anderen weitergegeben werden soll. Dabei k\"{o}nnen Knoten, die die Nachricht schon erhalten haben, auch als Zwischensender dienen, da dies eine energiesparendere Kommunikation erm\"{o}glicht. Gesucht sind nun konkrete Pfade, wie die einzelnen Knoten an den Startknoten angebunden werden.\\
Ein bisschen abstrakter haben wir einen vollst\"{a}ndigen gerichteten Graph $G = (V,E,d)$ gegeben. Eine L\"{o}sung wird nun durch einen zusammenh\"{a}ngenden gerichteten Spannbaum $T = (V,E_{T})$ repr\"{a}sentiert, in der die Gesamt\"{u}bertragungsleitung

\begin{displaymath} 
c(P) = \sum_{i \in V} \max_{(i,j) \in E_T} d_{ij}^l
\end{displaymath}

minimiert wird. In unserem Fall ist $l=3$. Ein Knoten $j$ bekommt nun die Nachricht von dem Knoten $i$, falls es eine Kante $(i,j) \in E_T$ gibt.

\subsection{Ant Colony Optimization}
Wir verwenden den \textit{MAX-MIN Ant System (MMAS)} als ACO Algorithmus. Dabei werden von 10 k\"{u}nstlichen Ameisen L\"{o}sungen generiert, die mittels einer lokalen Suche anschlie"send noch wenn m\"{o}glich verbessert werden. Das Pheromon-Modell sieht bei uns so aus, dass die Variable $\tau_{ij}$ den Wunsch vom Knoten $i$ zum Knoten $j$ die Nachricht zu senden ausdr\"{u}ckt. Zu Beginn und bei jedem Neustart nach einem bereits konvergierten Zustand werden alle $\tau_e = 0.5$ gesetzt. Beim Konstruieren einer L\"{o}sung verwendet jede Ameise eine eingeschr\"{a}nkte Kandidatenliste $E_{cand}$, von der sie \"{u}ber die Wahrscheinlichkeiten

\begin{displaymath} 
p(e) = \frac{\tau_{e} \cdot \eta(e)}{\sum_{e' \in E_{cand}} \tau_{e'} \cdot \eta(e')}
\end{displaymath}

der einzelnen m\"{o}glichen Kandidaten einen ausw\"{a}hlen. Erm\"{o}glicht dieser Schritt weitere Knoten ohne zus\"{a}tzliche Kosten der Teill\"{o}sungen hinzuzuf\"{u}gen, wird das anschlie"send erledigt. Danach wird der n\"{a}chste Knoten wieder mittels der Wahrscheinlichkeiten ausgew\"{a}hlt.\\
Nachdem alle Ameisen ihre L\"{o}sungen generiert haben, wird das Pheromon-Update \"{u}ber die Formel
 
\begin{displaymath} 
\tau_{e} = \min(\max(\tau_{min},\tau_{e} + p \cdot (\xi_e - \tau_{e})),\tau_{max})
\end{displaymath}

durchgef\"{u}hrt, wobei das $\xi_e=\frac{2}{3}$ falls $e$ in der besten L\"{o}sung aus dieser Iteration, $\xi_e=\frac{1}{3}$ falls
$e$ in der besten L\"{o}sung seit dem letzten Neustart oder $\xi_e=1$ falls $e$ in beiden besten L\"{o}sungen enthalten ist.

\subsection{Verbesserungsheuristik}
Als Verbesserungsheuristik wurde Variable Neighborhood Descent (VND) verwendet. Als Nachbarschaftsstruktur wurde r-shrink, das in Kapitel \ref{sec:shrink} beschrieben wird, verwendet. Dabei ist r die Anzahl der Nachkommen eines Knoten die versucht werden an einem anderen Knoten anzuh"angen. Im VND werden verschiedene Nachbarschaften verwendet, indem $r$ die Werte $1\ bis\ |V-1|$ annimmt.

\subsubsection{r-shrink}
\label{sec:shrink}
Hier wird f"ur einen Knoten $q$ versucht die $r$ am weitesten eintfernten Nachkommen $N$ von $q$ an einem anderen Knoten, des selben Levels $L(q)$, anzuh"angen. Ein Knoten $n \in N$ wird an einem anderen Knoten $p$ angeh"angt wenn $dist(q,n) > dist(p,n)$, wobei $dist(x,y)$ die Distanz zwischen Knoten $x$ und Knoten $y$ ist. 

\begin{table}[htbp]
\makebox[\linewidth]{
\begin{tabular}{cccccc}
\hline
Testinstanz & bester & durchschn. & Standard- & Laufzeitlimit[s] & Zeit bis zur\\
            & Zielfunktionswert & Zeilfunktionswert & abweichung &  & besten L�sung\\               
\hline
mebp-01 & 120.826.733 & 120.826.733 & 0 & 1 & 1s\\	
mebp-02 & 143.277.594 & 145.205.408 & 3.731.328 & 1 & 2s\\	
mebp-03 & 70.559.499 & 70.559.499 & 0 & 1 & 1s\\	
mebp-04 & 76.131.078 & 75.620.269 & 1.117.469 & 10 & 5:03 min\\														
mebp-05 & 74.330.081 & 76.297.278 & 2.860.131 & 10 & 4:28min\\
mebp-06 & 69.484.532 & 69.566.186 & 632.376 & 10 & 10s\\
mebp-07 & 28.801.520 & 29.007.928 & 142.955 & 250 & 1:56:40h\\
mebp-08 & 29.162.116 & 29.855.753 & 313.952 & 250 & 29:17min\\
mebp-09 & 27.592.543 & 28.093.996 & 213.241 & 250 & 1:59:04h\\
mebp-10 & 19.056.773 & 19.267.577 & 123.018 & 2000 & 13:56:34h\\
mebp-11 & 18.239.686 &  18.454.254 & 105.769 & 2000 & 03:53:38h\\
mebp-12 & 18.459.750 & 18.403.138 & 103.441 & 2000 & 31:01min\\
\hline
\end{tabular}
}
\end{table}

\end{document}
