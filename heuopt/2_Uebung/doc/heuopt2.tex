\documentclass[a4paper,10pt]{article}

\usepackage{ngerman}

%opening
\title{Heuristische Optimierungsverfahren \\ \bigskip \textbf{2.Programmier\"{u}bung}}
\author{Johannes Reiter - 0625101, Christian Gruber - 0625102}
\date{26.J\"{a}nner 2009}

\begin{document}

\maketitle

\section{Minimum Energy Broadcast Problem}
Das \textit{Minimum Energy Broadcast Problem (MEBP)} ist ein NP-hartes Optimierungsproblem f\"{u}r das Verbinden von mobilen Ger\"{a}ten in einem drahtlosen Ad-hoc Netzwerk. Es gilt eine Konfiguration f\"{u}r das Netzwerk zu finden, bei der die Sendeleitung und die n\"{o}tige Energie f\"{u}r die Kommunikation m\"{o}glichst minimal ist. Die Aufgabenstellung ist nun so, dass eine Nachricht ausgehend von einem vorgegebenen Knoten an alle anderen weitergegeben werden soll. Dabei k\"{o}nnen Knoten, die die Nachricht schon erhalten haben, auch als Zwischensender dienen, da dies eine energiesparendere Kommunikation erm\"{o}glicht. Gesucht sind nun konkrete Pfade, wie die einzelnen Knoten an den Startknoten angebunden werden.\\
Ein bisschen abstrakter haben wir einen vollst\"{a}ndigen gerichteten Graph $G = (V,E,d)$ gegeben. Eine L\"{o}sung wird nun durch einen zusammenh\"{a}ngenden gerichteten Spannbaum $T = (V,E_{T})$ repr\"{a}sentiert, in der die Gesamt\"{u}bertragungsleitung

\begin{displaymath} 
c(P) = \sum_{i \in V} \max_{(i,j) \in E_T} d_{ij}^l
\end{displaymath}

minimiert wird. In unserem Fall ist $l=3$. Ein Knoten $j$ bekommt nun die Nachricht von dem Knoten $i$, falls es eine Kante $(i,j) \in E_T$ gibt.

\subsection{Ant Colony Optimization}
Wir verwenden den \textit{MAX-MIN Ant System (MMAS)} als ACO Algorithmus. Dabei werden von 10 k\"{u}nstlichen Ameisen L\"{o}sungen generiert, die mittels einer lokalen Suche anschlie�end noch wenn m\"{o}glich verbessert werden. Das Pheromon-Modell sieht bei uns so aus, dass die Variable $\tau_{ij}$ den Wunsch vom Knoten $i$ zum Knoten $j$ die Nachricht zu senden ausdr\"{u}ckt. Zu Beginn und bei jedem Neustart nach einem bereits konvergierten Zustand werden alle $\tau_e = 0.5$ gesetzt. Beim Konstruieren einer L\"{o}sung verwendet jede Ameise eine eingeschr\"{a}nkte Kandidatenliste $E_{cand}$, von der sie \"{u}ber die Wahrscheinlichkeiten

\begin{displaymath} 
p(e) = \frac{\tau_{e} \cdot \eta(e)}{\sum_{e' \in E_{cand}} \tau_{e'} \cdot \eta(e')}
\end{displaymath}

der einzelnen m\"{o}glichen Kandidaten einen ausw\"{a}hlen. Erm\"{o}glicht dieser Schritt weitere Knoten ohne zus\"{a}tzliche Kosten der Teill\"{o}sungen hinzuzuf\"{u}gen, wird das anschlie�end erledigt. Danach wird der n\"{a}chste Knoten wieder mittels der Wahrscheinlichkeiten ausgew\"{a}hlt.\\
Nachdem alle Ameisen ihre L\"{o}sungen generiert haben, wird das Pheromon-Update \"{u}ber die Formel
 
\begin{displaymath} 
\tau_{e} = \min(\max(\tau_{min},\tau_{e} + p \cdot (\xi_e - \tau_{e})),\tau_{max})
\end{displaymath}

durchgef\"{u}hrt, wobei das $\xi_e=\frac{2}{3}$ falls $e$ in der besten L\"{o}sung aus dieser Iteration, $\xi_e=\frac{1}{3}$ falls
$e$ in der besten L\"{o}sung seit dem letzten Neustart oder $\xi_e=1$ falls $e$ in beiden besten L\"{o}sungen enthalten ist.

\subsection{Verbesserungsheuristik}


\subsection{Vergleich der verschiedenen Varianten f\"{u}r die lokale Suche}

\begin{tabular}{cccc}
\hline
Testinstanz & Nachbarschaft & Schrittfunktion & Mittelwert \\
\hline
$4\zeta_{10}^{10}$ & move & random &  12.9 \\
							 		& move & next &  11.2 \\
									& move & best &  11.2 \\
							 		& 2-exchange & random &  13.2 \\
							 		& 2-exchange & next &  11.1 \\
									& 2-exchange & best &  11.6 \\
							 		& split & random &  13.1 \\
							 		& split & next &  11.6 \\
									& split & best &  11.6 \\
							 		& rotate subsequence & random &  12.8 \\
							 		& rotate subsequence & next &  11.3 \\
									& rotate subsequence & best &  11.2 \\
\hline									
$15\zeta_{40}^{30}$ & move & random &  157.6 \\
							 		& move & next &  119.6 \\
									& move & best &  119.5 \\
							 		& 2-exchange & random &  158 \\
							 		& 2-exchange & next &  122.7 \\
									& 2-exchange & best &  124 \\
							 		& split & random &  158.6 \\
							 		& split & next &  134.2 \\
									& split & best &  133.8 \\
							 		& rotate subsequence & random &  158.1 \\
							 		& rotate subsequence & next &  118.9 \\
									& rotate subsequence & best &  118.5 \\
\hline									
\end{tabular}

\begin{tabular}{cccc}
\hline
Testinstanz & Nachbarschaft & Schrittfunktion & Mittelwert \\
\hline									
$20\zeta_{60}^{40}$ & move & random &  283.2 \\
							 		& move & next &  196.2 \\
									& move & best &  196.5 \\
							 		& 2-exchange & random &  283.8 \\
							 		& 2-exchange & next &  203.8 \\
									& 2-exchange & best &  203.5 \\
							 		& split & random &  284.5 \\
							 		& split & next &  228.8 \\
									& split & best & 227.5 \\
							 		& rotate subsequence & random &  281.2 \\
							 		& rotate subsequence & next &  192.8 \\
									& rotate subsequence & best & 192.2 \\																																																		
\hline
\end{tabular}

\bigskip

\subsection{Variable Neighborhood Descent (VND)}
Wir haben mit der VND jeweils 30 runs gemacht und als Schrittfunktion ist best verwendet worden. Sehr sch\"{o}n kann man hier erkennen, wie man aus einem lokalen Optimum f\"{u}r eine Nachbarschaft durch einen Wechsel der Nachbarschaftsstruktur wieder heraus finden kann und dadurch das Ergebnis weiter verbessert wird.
\bigskip

\begin{tabular}{cccc}
\hline
Testinstanz & Mittelwert & Bestes Ergebnis & Standardabweichung \\
\hline
$4\zeta_{10}^{10}$ & 11 & 10 & 0.8 \\									
$15\zeta_{40}^{30}$ & 117.2 & 114 & 2.3 \\									
$20\zeta_{60}^{40}$ & 193.9 & 188 & 4 \\						
\hline
\end{tabular}

\bigskip

\subsection{Generalized Variable Neighborhood Search (GVNS)}

Wir haben mit der GVNS jeweils 30 runs gemacht und als Schrittfunktion haben wir best verwendet. Man kann aus den erzielten Ergebnissen relativ eindeutig die Verbesserung durch die GVNS erkennen. In den log-Dateien sieht man auch wie sehr dieses "`shaking"' helfen kann, aus einem lokalen Optimum wieder heraus zu finden.
\bigskip

\begin{tabular}{cccc}
\hline
Testinstanz & Mittelwert & Bestes Ergebnis & Standardabweichung \\
\hline
$4\zeta_{10}^{10}$ & 10.5 & 10 & 0.5 \\									
$15\zeta_{40}^{30}$ & 113.7 & 111 & 2 \\									
$20\zeta_{60}^{40}$ & 185.2 & 181 & 3.0 \\						
\hline
\end{tabular}

\end{document}
