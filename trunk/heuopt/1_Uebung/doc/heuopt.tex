\documentclass[a4paper,10pt]{article}

\usepackage{ngerman}

%opening
\title{Heuristische Optimierungsverfahren \\ \bigskip \textbf{1.Programmier\"{u}bung}}
\author{Johannes Reiter, Christian Gruber}
\date{01.Dezember 2009}

\begin{document}

\maketitle

\section{Tool Switching Problem}
Gegeben ist eine Menge von Jobs, die zu ihrer Abarbeitung jeweils einen gewissen Satz von Tools ben\"{o}tigt.
Da nicht alle Tools gemeinsam in das Magazin der Maschine passen (max. Anzahl entspricht der Magazingr\"{o}"se), ist
nun eine m\"{o}glichst g\"{u}nstige Reihenfolge der Jobs gesucht, in der die Anzahl der Tool Switches minimiert wurde.
Die L\"{o}sung zu einer Test-Instanz eines Tool Switching Problems ist nun eine Permutation in der alle Jobs abgearbeitet
werden und eine dazu passende Sequenz von Magazin-Konfigurationen.

\subsection{Tool-Konfiguration f\"{u}r eine vorgegebene Reihenfolge}
Gesucht ist hier eine m\"{o}glichst optimale Magazin-Konfiguration f\"{u}r eine vorgegebene Reihenfolge f\"{u}r die Abarbeitung der Jobs. In diesem Spezialfall kann man nun ausnutzen, dass der Algorithmus genau wei"s wann ein Tool das n\"{a}chste Mal ben\"{o}tigt wird und somit zu der vorgegebenen Reihenfolge die optimale Tool-Konfiguration erzeugen kann.

\subsection{Konstruktionsheuristik}
Unsere Konstruktionsheuristik w\"{a}hlt am Beginn zuf\"{a}llig den ersten Job aus. Die weiteren Jobs werden so gew\"{a}hlt, dass m\"{o}glichst wenig Kosten entstehen. D.h. der Algorithmus sieht sich alle noch offenen Jobs an und vergleicht die ben\"{o}tigten Tools mit der aktuellen Magazin-Konfiguration.

\subsection{Nachbarschaftsstrukturen}
\begin{itemize}
	\item \textbf{move:} In dieser Nachbarschaft wird ein Job ausgew\"{a}lt und dieser zuf\"{a}llig an eine andere Position in der Sequenz verschoben. 
	\item \textbf{2-exchange (switch):} Beim 2-exchange tauschen 2 Jobs einfach ihre Position in der Sequenz aus.
	\item \textbf{split and rotate:} Hier wird die Permutation von Jobs geteilt und die Reihenfolge der beiden Teilsequenzen ausgetauscht.
	\item \textbf{rotate in subsequence:} Es wird irgendeine Teilfolge der Sequenz ausgew\"{a}hlt, diese wird wiederum geteilt und die entstanden Subsequenzen werden dann rotiert.
\end{itemize}

F\"{u}r die vier verschiedenen Nachbarschaftsstrukturen haben wir jeweils auch die drei Schrittfunktionen random neighbor, next improvement und best improvement implementiert. Eine inkrementelle Bestimmung der Zielfunktionswerte von Nachbarl\"{o}ungen w\"{a}re zwar m\"{o}glich gewesen, jedoch nicht besonders sinnvoll, da man mit dem im 1.Punkt implementierten Algorithmus die jeweils optimale Tool-Konfiguration f\"{u}r eine vorgegebene Reihenfolge von Jobs bekommt und das Ergebnis der Kostenfunktion eigentlich erst dann wirklich aussagekr\"{a}ftig ist.

\subsection{Vergleich der verschiedenen Varianten f\"{u}r die lokale Suche}

\begin{tabular}{cccccc}
\hline
Testinstanz & Nachbarschaft & Schrittfunktion & Mittelwert \\
\hline
$4\zeta_{10}^{10}$ & move & random &  \\
							 		& move & next &  \\
									& move & best &  \\
							 		& 2-exchange & random &  \\
							 		& 2-exchange & next &  \\
									& 2-exchange & best &  \\
							 		& split & random &  \\
							 		& split & next &  \\
									& split & best &  \\
							 		& rotate subsequence & random &  \\
							 		& rotate subsequence & next &  \\
									& rotate subsequence & best &  \\
\hline									
$15\zeta_{40}^{30}$ & move & random &  \\
							 		& move & next &  \\
									& move & best &  \\
							 		& 2-exchange & random &  \\
							 		& 2-exchange & next &  \\
									& 2-exchange & best &  \\
							 		& split & random &  \\
							 		& split & next &  \\
									& split & best &  \\
							 		& rotate subsequence & random &  \\
							 		& rotate subsequence & next &  \\
									& rotate subsequence & best &  \\
\hline									
$20\zeta_{60}^{40}$ & move & random &  \\
							 		& move & next &  \\
									& move & best &  \\
							 		& 2-exchange & random &  \\
							 		& 2-exchange & next &  \\
									& 2-exchange & best &  \\
							 		& split & random &  \\
							 		& split & next &  \\
									& split & best &  \\
							 		& rotate subsequence & random &  \\
							 		& rotate subsequence & next &  \\
									& rotate subsequence & best &  \\																																																		
\hline
\end{tabular}

\subsection{Variable Neighborhood Descent (VND)}

\subsection{Generalized Variable Neighborhood Search (GVNS)}

Wir haben mit der GVNS jeweils 30 runs gemacht und als Schrittfunktion haben wir best verwendet.

\begin{tabular}{cccc}
\hline
Testinstanz & Mittelwert & Bestes Ergebnis & Standardabweichung \\
\hline
$4\zeta_{10}^{10}$ & 10.5 & 10 & 0.5 \\									
$15\zeta_{40}^{30}$ & 113.7 & 111 & 2 \\									
$20\zeta_{60}^{40}$ & 186.7 & 182 & 2.9 \\						
\hline
\end{tabular}

\end{document}
